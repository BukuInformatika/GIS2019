\section{ Pengertian MapServer}

MapServer adalah sebuah aplikasi pengembangan yang bersifat terbuka (open source) untuk pengembangan aplikasi internet yang melakukan pengolahan spasial. Bisa dijalankan sebagai sebuah program CGI atau Mapscript yang mendukung beberapa bahasa pemrograman.
MapServer adalah aplikasi Open Source yang memungkinkan suatu data peta diakses melalui web. Teknologi mapserver pertama kali dikembangkan oleh Universitas Minesotta Amerika Serikat. Dengan adanya MapServer menjadikan pekerjaan membuat Peta Digital menjadi lebih mudah dan interaktif. Maksud dari Interaktif peta disini diartikan bahwa user dapat dengan mudah mengubah dan melihat tampilan peta seperti memperbesar atau memperkecil gambar, rotate, dan menampilkan informasi (seperti menampilkan info jalan) dan analisis pada permukaan geografi. 
MapServer merupakan sebuah program aplikasi GIS berbasis web yang open source. MapServer juga dikembangkan tanpa tujuan komersial, sehingga pengguna MapServer dapat menggunakan dan mengembangkan program MapServer. 
Mapserver merupakan aplikasi freeware dan open source yang meungkinkan kita menampilkan dataspasial (peta) dalam platform web.Aplikasi ini pertama kali dikembangkan oleh Universitas Minesota, Amerika Serikat untuk proyekFor Net (sebuah proyek untuk sumber daya alam )yang disponsori oleh NASA (National Aeronauticsand Space Administration) Support NASA dilanjutkandan dikembangkan proyek TerraSIP  untuk manajemendata lahan. Saat ini sifatnya yang terbuka (open source), pengembangan suatu mapserver dilakukan oleh pengembang dari berbagai Negara. 
Pada bentuk yang paling dasar (based), MapServer berupa sebuah  program CGI (Common Gateway Interface). Program Mapserver tersebut dieksekusi pada sebuah webserver dengan konfigurasi peta yang disimpan dalam sebuah file \*.MAP, kemudian kirim dan ditampilkan oleh web browser baik dalam bentuk gambar peta atau  bentuk yang lain. 
