\section{Mapfile}
Mapfile adalah jantung dari MapServer. Mapfile mendefinisikan hubungan antara objek, mendefinisikan tempat data berada dan menentukan bagaimana hal-hal yang harus digambar. 

File .MAP adalah file teks ASCII, dan terdiri dari berbagai objek. Setiap objek memiliki beragam parameter yang tersedia. Mapfile terdiri dari objek MAP, yang harus dimulai dengan kata MAP. Konsep LAPISAN, Lapisan adalah kombinasi data dan model. Data, dalam bentuk atribut geometri, diberi model menggunakan arahan CLASS dan STYLE.

Mapfile menyediakan antarmuka untuk MapServer untuk pembuatan aplikasi Web. Mapfile dapat digunakan secara independen dari CGI MapServer atau modulnya dapat dimuat yang menambahkan MapServer ke bahasa PHP, Perl, Python, Ruby, Tcl, Java, dan .NET.

Contoh mapfile sederhana yang hanya menampilkan satu layer serta output gambar peta.
\begin{lstlisting}
MAP
    NAME "sample"
    STATUS ON
    SIZE 600 400
    SYMBOLSET "../etc/symbols.txt"
    EXTENT -180 -90 180 90
    UNITS DD
    SHAPEPATH "../data"
    IMAGECOLOR 255 255 255
    FONTSET "../etc/fonts.txt"
    # Start of web interface definition
    WEB
        IMAGEPATH "/ms4w/tmp/ms_tmp/"
        IMAGEURL "/ms_tmp/"
    END # WEB
    # Start of layer definitions
    LAYER
        NAME 'global-raster'
        TYPE RASTER
        STATUS DEFAULT
        DATA bluemarble.gif
    END # LAYER
END # MAP
\end{lstlisting}

Catatan:
\begin{enumerate}
\item Komentar dalam mapfile ditentukan dengan karakter '\#'.
\item MapServer mem-parsing mapfile dari atas ke bawah, oleh karena itu lapisan pada ujung mapfile akan ditarik terakhir.
\item Path harus dikutip (kutipan tunggal atau ganda).
\item Mapfile tidak case-sensitive.
\item String yang berisi karakter non-alfanumerik atau kata kunci MapServer HARUS dikutip. Disarankan untuk menempatkan SEMUA string dalam tanda kutip ganda.
\item Mapfile memiliki struktur hierarkis, dengan objek MAP menjadi "root".
\end{enumerate}