\section{Definisi GIS} 
Sistem Informasi Geografis atau disingkat SIG dalam bahasa Inggris Geographic Information System (disingkat GIS) merupakan sistem informasi khusus yang mengelola data yang memiliki informasi spasial (bereferensi keruangan). Atau dalam arti yang lebih sempit adalah sistem komputer yang memiliki kemampuan untuk membangun, menyimpan, mengelola dan menampilkan informasi bereferensi geografis atau data geospasial untuk mendukung pengambilan keputusan dalam perencanaan dan pengelolaan suatu wilayah, misalnya data yang diidentifikasi menurut lokasinya, dalam sebuah database. Para praktisi juga memasukkan orang yang membangun dan mengoperasikannya dan data sebagai bagian dari sistem ini. 
Teknologi Sistem Informasi Geografis dapat digunakan untuk investigasi ilmiah, pengelolaan sumber daya, perencanaan pembangunan, kartografi dan perencanaan rute. Misalnya, SIG bisa membantu perencana untuk secara cepat menghitung waktu tanggap darurat saat terjadi bencana alam, atau SIG dapat digunaan untuk mencari lahan basah (wetlands) yang membutuhkan perlindungan dari polusi atau dapat digunakan mencari informasi sebuah tempat khusus dan banyak manfaat lain yang dapat ikembangkan dalam sistem informasi geografis ini. 

\subsection{Pengertian GIS Menurut Para Ahli}
\begin{enumerate}

\item \textbf{Aronaff (1989)}
SIG adalah sistem informasi yang didasarkan pada kerja komputer yang memasukkan, mengelola, memanipulasi dan menganalisis data serta memberi uraian.

\item \textbf{Burrough (1986)}
SIG merupakan alat yang bermanfaat untuk pengumpulan, penimbunan, pengambilan kembali data yang diinginkan dan penayangan data keruangan yang berasal dari kenyataan dunia.

\item \textbf{Murai (1999)}
SIG sebagai sistem informasi yang digunakan untuk memasukkan, menyimpan, memanggil kembali, mengolah, menganalisis dan menghasilkan data bereferensi geografis atau data geospasial untuk mendukung pengambilan keputusan dalam perencanaan dan pengelolaan penggunaan lahan, sumber daya alam, lingkungan, transportasi, fasilitas kota, dan pelayanan umum lainnya.

\item \textbf{Marble et al (1983)}
SIG merupakan sistem penanganan data keruangan.  Bernhardsen (2002) SIG sebagai sistem komputer ang digunakan untuk memanipulasi data geografi. Sistem ini diimplementasikan dengan perangkat keras dan perangkat lunak komputer yang berfungsi untuk akusisi dan verifikasi data, kompilasi data, penyimpanan data, perubahan dan pembaharuan data, manajemen dan pertukaran data, manipulasi data, pemanggilan dan presentasi data serta analisa data.

\item \textbf{Gistut (1994)}
SIG adalah sistem yang dapat mendukung pengambilan keputusan spasial dan mampu mengintegrasikan deskripsi-deskripsi lokasi dengan karakteristik-karakteristik fenomena yang ditemukan di lokasi tersebut. SIG yang lengkap mencakup metodologi dan teknologi yang diperlukan, yaitu data spasial perangkat keras, perangkat lunak dan struktur organisasi  Berry (1988) SIG merupakan sistem informasi, referensi internal, serta otomatisasi data keruangan.

\item \textbf{Calkin dan Tomlison (1984)}
SIG merupakan sistem komputerisasi data yang penting.  Linden, (1987) SIG adalah sistem untuk pengelolaan, penyimpanan, pemrosesan (manipulasi), analisis dan penayangan data secara spasial terkait dengan muka bumi.

\item \textbf{Alter}
SIG adalah sistem informasi yang mendukung pengorganisasian data, sehingga dapat diakses dengan menunjuk daerah pada sebuah peta.

\item \textbf{Prahasta}
SIG merupakan sejenis software yang dapat digunakan untuk pemasukan, penyimpanan, manipulasi, menampilkan, dan keluaran informasi geografis berikut atribut-atributnya.  Petrus Paryono SIG adalah sistem berbasis komputer yang digunakan untuk menyimpan, manipulasi dan menganalisis informasi geografi. Dari definisi-definisi dari para ahli di atas dapat disimpulkan bahwa SIG merupakan pengelolaan data geografis yang didasarkan pada kerja komputer (mesin).

\item \textbf{Petrus Paryono}
SIG adalah sistem berbasis komputer yang digunakan untuk menyimpan, manipulasi dan menganalisis informasi geografi. Dari definisi-definisi dari para ahli di atas dapat disimpulkan bahwa SIG merupakan pengelolaan data geografis yang didasarkan pada kerja komputer (mesin).

\item \textbf{Chrisman (1997)}
SIG adalah sistem yang terdiri dari perangkat keras , perangkat lunak , data,manusia (brainware) organisasi dan lembaga yang digunakan untuk mengumpulkan,menyimpan,menganalisis dan menyebarkan informasi informasi mengenai daerah daerah di permukaan bumi.

\item \textbf{Lukman (1993)}
menyatakan bahwa sistem informasi geografi menyajikan informasi keruangan beserta atributnya yang terdiri dari beberapa komponen utama yaitu:
\begin{enumerate}
\item Masukan data merupakan proses pemasukan data pada komputer dari peta (peta topografi dan peta tematik), data statistik, data hasil analisis penginderaan jauh data hasil pengolahan citra digital penginderaan jauh, dan lain-lain. Data-data spasial dan atribut baik dalam bentuk analog maupun data digital tersebut dikonversikan kedalam format yang diminta oleh perangkat lunak sehingga terbentuk basisdata (database). Menurut Anon (2003) basisdata adalah pengorganisasian data yang tidak berlebihan dalam komputer sehingga dapat dilakukan pengembangan, pembaharuan, pemanggilan, dan dapat digunakan secara bersama oleh pengguna.

\item Penyimpanan data dan pemanggilan kembali (data storage dan retrieval) ialah penyimpanan data pada komputer dan pemanggilan kembali dengan cepat (penampilan pada layar monitor dan dapat ditampilkan/cetak pada kertas).

\item Manipulasi data dan analisis ialah kegiatan yang dapat dilakukan berbagai macam perintah misalnya overlay antara dua tema peta, membuat buffer zone jarak tertentu dari suatu area atau titik dan sebagainya. Anon (2003) mengatakan bahwa manipulasi dan analisis data merupakan ciri utama dari SIG. Kemampuan SIG dalam melakukan analisis gabungan dari data spasial dan data atribut akan menghasilkan informasi yang berguna untuk berbagai aplikasi.

\item Pelaporan data ialah dapat menyajikan data dasar, data hasil pengolahan data dari model menjadi bentuk peta atau data tabular. Menurut Barus dan wiradisastra (2000) Bentuk produk suatu SIG dapat bervariasi baik dalam hal kualitas, keakuratan dan kemudahan pemakainya. Hasil ini dapat dibuat dalam bentuk peta-peta, tabel angka-angka: teks di atas kertas atau media lain (hard copy), atau dalam cetak lunak (seperti file elektronik).
\end{enumerate}

\item \textbf{Menurut Anon (2003)}
 ada beberapa alasan mengapa perlu menggunakan SIG, diantaranya adalah:
\begin{enumerate}
\item SIG menggunakan data spasial maupun atribut secara terintegrasi.
\item SIG dapat digunakansebagai alat bantu interaktif yang menarik dalam usaha meningkatkan pemahaman mengenai konsep lokasi, ruang, kependudukan, dan unsur-unsur geografi yang ada dipermukaan bumi.
\item SIG dapat memisahkan antara bentuk presentasi dan basis data.
\item SIG memiliki kemampuan menguraikan unsur-unsur yang ada dipermukaan bumi kedalam beberapa layer atau coverage data spasial
\item SIG memiliki kemapuan yang sangat baik dalam memvisualisasikan data spasial berikut atributnya
\item Semua operasi SIG dapat dilakukan secara interaktif
\item SIG dengan mudah menghsilkan peta-peta tematik
\item semua operasi SIG dapat di costumize dengan menggunakan perintah-perintah dalam bahaa script.
\item Peragkat lunak SIG menyediakan fasilitas untuk berkomunikasi dengan perangkat lunak lain
\item SIG sangat membantu pekerjaan yang erat kaitannya dengan bidang spasial dan geoinformatika.
\end{enumerate}
\end{enumerate}