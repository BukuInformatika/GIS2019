<<<<<<< HEAD
\section{SUMBER-SUMBER DATA GEOSPASIAL}

Istilah geospatial data dapat juga diganti dengan spatial data atau data GIS (geospatial information system data) adalah data tentang aspek fisik dan administratif dari sebuah objek geografis. Aspek fisik di sini mencakup pula bentuk anthropogenic dan bentuk alam baik yang terdapat di permukaan maupun di bawah permukaan bumi. Bentuk anthropogenic mengandung di dalamnya fenomena seperti jalan, rel kereta api, bangunan, jembatan, dan sebagainya. Juga terdapat bentuk alam tertentu saja yakni sungai, danau, pantai, daratan tinggi, dan sebagainya. Sedangkan aspek administratif adalah pembagian atau pembatasan sosio-kultural yang dibuat oleh suatu organisasi atau badan untuk keperluan pengaturan dan pemakaian sumberdaya alam. Termasuk dalam aspek administratif ini adalah batas negara, pembagian wilayah administrasi, zona, kode pos, batas kepemilikan tanah, dan sebagainya. 

Sumber informasi tercetak yang dianjurkannya adalah “GIS Data Sources” karangan Decker (terbitan John Wiley \& Sons, 2001) dan “GIS and Public Data” karangan Ralston (terbitan Dalmar Learning, 2004). Beberapa portal dan clearing house internasional juga menyediakan informasi tentang sumber-sumber data GIS, misalnya:
\begin{enumerate}
\item Geospatial One-Stop
\item National Spatial Data Clearinghouse
\item GIS Data Depo
\item Geography Network
\item USGS EROS Data Center
\item The National Map
\item NGA Geospatial Engine
\item The Harvard Geospatial Library
\item Alexandria Digital Library
\item Global Land Cover Facility
\end{enumerate}

Sementara penjaja data swasta internasional yang dianggap populer adalah:
\begin{enumerate}
\item ESRI
\item East View Cartographic
\item Map Mart
\item GfK Macon
\item GIS Data Depot
\item LAND INFO
\item LeadDog Consulting
\item Collins-Bartholomew
\item ACASIAN
\item Digital Globe
\item GeoEye
\item MapInfo
\end{enumerate}

=======
\section{Membuat Data Vektor}
Disusun oleh:

Eko cahyono putro 1164035
Nur Arkhamia Batubara 1164049

\subsection {Pengertian Data Vektor}
Data vektor merupakan tipe data yang umum ditemukan dalam SIG. Sebuah vektor pada intinya merupakan sesuatu yang berbentuk sebuah titik, atau garis yang menghubungkan titik-titik tersebut. Dengan kata lain, titik, garis, dan poligon merupakan vektor (garis lengkung merupakan vektor juga).

Salah satu hal yang penting untuk dicatat adalah \textit{layer} QGIS hanya mengandung satu tipe fitur. Artinya, satu layer tidak dapat mengandung fitur titik dan fitur garis, karena mereka merupakan tipe data yang berbeda. Namun apabila anda ingin memiliki sebuah \textit{file} yang memiliki \textit{polygon} sekolah dan file lain yang memiliki titik-titik sekolah, anda dapat menambahkan mereka sebagai dua \textit{layer} yang terpisah\cite{setiawan2018membuka}.
>>>>>>> 2b3dcb1e94f029a00005cbe9dfb71b18b6c56f48
