
\section{SUMBER-SUMBER DATA GEOSPASIAL}

Istilah geospatial data dapat juga diganti dengan spatial data atau data GIS (geospatial information system data) adalah data tentang aspek fisik dan administratif dari sebuah objek geografis. Aspek fisik di sini mencakup pula bentuk anthropogenic dan bentuk alam baik yang terdapat di permukaan maupun di bawah permukaan bumi. Bentuk anthropogenic mengandung di dalamnya fenomena seperti jalan, rel kereta api, bangunan, jembatan, dan sebagainya. Juga terdapat bentuk alam tertentu saja yakni sungai, danau, pantai, daratan tinggi, dan sebagainya. Sedangkan aspek administratif adalah pembagian atau pembatasan sosio-kultural yang dibuat oleh suatu organisasi atau badan untuk keperluan pengaturan dan pemakaian sumberdaya alam. Termasuk dalam aspek administratif ini adalah batas negara, pembagian wilayah administrasi, zona, kode pos, batas kepemilikan tanah, dan sebagainya. 

Sumber informasi tercetak yang dianjurkannya adalah “GIS Data Sources” karangan Decker (terbitan John Wiley \& Sons, 2001) dan “GIS and Public Data” karangan Ralston (terbitan Dalmar Learning, 2004). Beberapa portal dan clearing house internasional juga menyediakan informasi tentang sumber-sumber data GIS, misalnya:
\begin{enumerate}
\item Geospatial One-Stop
\item National Spatial Data Clearinghouse
\item GIS Data Depo
\item Geography Network
\item USGS EROS Data Center
\item The National Map
\item NGA Geospatial Engine
\item The Harvard Geospatial Library
\item Alexandria Digital Library
\item Global Land Cover Facility
\end{enumerate}

Sementara penjaja data swasta internasional yang dianggap populer adalah:
\begin{enumerate}
\item ESRI
\item East View Cartographic
\item Map Mart
\item GfK Macon
\item GIS Data Depot
\item LAND INFO
\item LeadDog Consulting
\item Collins-Bartholomew
\item ACASIAN
\item Digital Globe
\item GeoEye
\item MapInfo
\end{enumerate}

\subsection{Pendahuluan}
Dalam dunia geospasial tidak jauh dengan data spasial. Data spasial ibarat hokum mutlak diperlukan dalam membuat peta atau melakukan analisis spasial. Namun kendalanya tidak semua data-data yang diperlukan tersedia. Dalam uraian berikut akan membahas sumber data spasial dari open geodata.

\subsection{Ina Geoportal}
Ina Geoportal adalah sumber data geospasial resmi untuk Indonesia yang  dibangun, dipelihara dan diawasi langsung oleh Badan Informasi Geospasial (BIG) yang di mana merupakan lembaga pemerintah yang bertanggung jawab penuh atas data geospasial nasional. Melalui Ina Geoportal ini kita dapat men-download data-data peta rupa bumi dalam skala 250 ribu, 50 ribu dan 25 ribu. Proses mendapatkan datanya pun cukup mudah, Kita hanya perlu mengisikan Nama, email, jenis data RBI, jenis pengguna dan terakhir tentu saja kita harus menyetujui ketentuan undang-undang yang berlaku. 
\begin{figure}[htbp]
\centering
\includegraphics[width=1\textwidth]{pictures/ina_geospasial}
\caption{Ina Geoportal}
\label{labelgambar1}
\end{figure}

\subsection{USGS Earth Explorer}
USGS earth explorer merupakan sumber data spasial yang disediakan oleh lembaga survey geologi Amerika Serikat. Di earth explorer ini disediakan cukup banyak sekali data dengan berbagai macam tema, resolusi dan sensor, seperti citra satelit, Lidar, cuaca, radar, landcover dan lain sebagainya. Data-data tersedia umumnya mencakup data di wilayah Amerika. Namun, tidak hanya data-data tersebut yang tersedia melainkan data-data dengan cakupan global seperti data Digital Elevation Model (DEM), SRTM, citra satelit Landsat, monitoring vegetasi dan lain-lain.
\begin{figure}[htbp]
\centering
\includegraphics[width=1\textwidth]{pictures/usgs_earth_explorer}
\caption{USGS Earth Explorer}
\label{labelgambar2}
\end{figure}

\subsection{Worldclim}
Worldclim adalam sumber data geospasial yang menyediakan data curah hujan guna melakukan proses analisis spasial yang tersedia dalam format spasial. Worldclim menyuguhkan data curah hujan dan data iklim secara umum yang meliputi temperatur tahunan serta bulanan. Data ini diperoleh dari stasiun-stasiun cuaca di seluruh dunia yang dikumpulkan jadi satu dari tahun 1960-1990 (versi 1.4) dan 1970-2000 (versi 2). Data-data yang dikumpulkan kemudian diolah dan dianalisa sehingga dapat diprediksi data iklim untuk masa lalu, sekarang dan masa yang akan datang. Jadi, data ini bukan termasuk data realtime, akan tetapi analisa data iklim selama 30 tahun. Data wordclim dapat diperoleh dalam format raster dengan resolusi 1 km. 
\begin{figure}[htbp]
\centering
\includegraphics[width=1\textwidth]{pictures/data_worldclim}
\caption{USGS Data Worldclim}
\label{labelgambar3}
\end{figure}


