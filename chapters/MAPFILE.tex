\section{Mapfile}
Mapfile adalah jantung dari MapServer. Mapfile mendefinisikan hubungan antara objek, mendefinisikan tempat data berada dan menentukan bagaimana hal-hal yang harus digambar. 

File .MAP adalah file teks ASCII, dan terdiri dari berbagai objek. Setiap objek memiliki beragam parameter yang tersedia. Mapfile terdiri dari objek MAP, yang harus dimulai dengan kata MAP. Konsep LAPISAN, Lapisan adalah kombinasi data dan model. Data, dalam bentuk atribut geometri, diberi model menggunakan arahan CLASS dan STYLE.

Mapfile menyediakan antarmuka untuk MapServer untuk pembuatan aplikasi Web. Mapfile dapat digunakan secara independen dari CGI MapServer atau modulnya dapat dimuat yang menambahkan MapServer ke bahasa PHP, Perl, Python, Ruby, Tcl, Java, dan .NET.

Contoh mapfile sederhana yang hanya menampilkan satu layer serta output gambar peta.
\begin{lstlisting}
MAP
    NAME "sample"
    STATUS ON
    SIZE 600 400
    SYMBOLSET "../etc/symbols.txt"
    EXTENT -180 -90 180 90
    UNITS DD
    SHAPEPATH "../data"
    IMAGECOLOR 255 255 255
    FONTSET "../etc/fonts.txt"
    # Start of web interface definition
    WEB
        IMAGEPATH "/ms4w/tmp/ms_tmp/"
        IMAGEURL "/ms_tmp/"
    END # WEB
    # Start of layer definitions
    LAYER
        NAME 'global-raster'
        TYPE RASTER
        STATUS DEFAULT
        DATA bluemarble.gif
    END # LAYER
END # MAP
\end{lstlisting}

Catatan:
\begin{enumerate}
\item Komentar dalam mapfile ditentukan dengan karakter '\#'.
\item MapServer mem-parsing mapfile dari atas ke bawah, oleh karena itu lapisan pada ujung mapfile akan ditarik terakhir.
\item Path harus dikutip (kutipan tunggal atau ganda).
\item Mapfile tidak case-sensitive.
\item String yang berisi karakter non-alfanumerik atau kata kunci MapServer HARUS dikutip. Disarankan untuk menempatkan SEMUA string dalam tanda kutip ganda.
\item Mapfile memiliki struktur hierarkis, dengan objek MAP menjadi "root".
\end{enumerate}

\section{Pembahasan Syntax}
\subsection{MAP}
Objek peta dimulai dengan kata MAP, dan berakhir dengan kata END.
\begin{lstlisting}
MAP
  NAME "sample"
  EXTENT -180 -90 180 90 # Geographic
  SIZE 800 400
  IMAGECOLOR 128 128 255
END #MAP
\end{lstlisting}

Penjelasan:
\begin{enumerate}
\item NAME adalah nama map
\item EXTENT adalah tingkat output dalam unit peta output.
\item SIZE adalah lebar dan tinggi gambar peta dalam piksel.
\item IMAGECOLOR adalah warna latar belakang gambar default.
\end{enumerate}

\subsection{LAYER Object}
Objek layer ada 2 yaitu Raster dan Vektor.
\begin{enumerate}
\item Raster Layers
\begin{lstlisting}
LAYER
    NAME "bathymetry"
    TYPE RASTER
    STATUS DEFAULT
    DATA "bath_mapserver.tif"
END # LAYER
\end{lstlisting}

Penjelasan:
\begin{enumerate}
\item NAME adalah nama layer.
\item TYPE adalah jenis dari layer tsb.
\item STATUS adalah kondisi dari layer tsb.
\item DATA adalah file SHP atau tif yang berisi informasi koordinat.
\end{enumerate}

\item Vertor Layer
\begin{lstlisting}
LAYER
  NAME "world_poly"
  DATA 'shapefile/countries_area.shp'
  STATUS ON
  TYPE POLYGON
  CLASS
    NAME 'The World'
    STYLE
      OUTLINECOLOR 0 0 0
    END # STYLE
  END # CLASS
END # LAYER
\end{lstlisting}

Penjelasan:
\begin{enumerate}
\item NAME adalah nama layer.
\item TYPE adalah jenis dari layer yang akan di tampilkan seperti point, line, atau polygon.
\item STATUS adalah kondisi dari layer tsb.
\item DATA adalah file SHP atau tif yang berisi informasi koordinat
\end{enumerate}
\end{enumerate}

\subsection{CLASS and STYLE Objects}
\begin{lstlisting}
CLASS
  NAME "Primary Roads"
  STYLE
    SYMBOL "circle"
    COLOR 178 114 1
    SIZE 15
  END # STYLE
  STYLE
    SYMBOL "circle"
    COLOR 254 161 0
    SIZE 7
  END # STYLE
END # CLASS
\end{lstlisting}

Penjelasan:
\begin{enumerate}
\item NAME adalah nama layer.
\item STYLE memegang parameter untuk simbolisasi dan gaya. Berbagai gaya dapat diterapkan dalam CLASS atau LABEL.
\item COLOR penentuan warna pada objek.
\end{enumerate}

\subsection{PROJECTION}
Untuk mengatur proyeksi Anda harus menetapkan satu objek proyeksi untuk gambar output (dalam objek MAP) dan satu objek proyeksi untuk setiap lapisan (dalam objek LAPISAN) yang akan diproyeksikan. Objek proyeksi terdiri dari serangkaian kata kunci PROJ.4, yang dapat ditentukan dalam objek secara langsung atau dirujuk dalam file EPSG. File EPSG adalah file pencarian yang berisi parameter proyeksi, dan merupakan bagian dari perpustakaan PROJ.4.

Contoh parameter proyeksi:
\begin{lstlisting}
PROJECTION
   "init=epsg:4326"
END
\end{lstlisting}

\section{Tutorial Membuat mapfile untuk WCS, WFS, WMS dan WMTS}
\subsection{Membuat Mapfile untuk WCS}
\begin{enumerate}
\item Buat file baru dengan nama wcs.map untuk menyiman script konfigurasi wcs
\item Masukan script map objek di bawah ini ke file wcs.map, pada script ini kita menentukan lebar peta, warna gambar serta mendefinisikan sumber map, fontset dan symbolset yang akan digunakan. 
\begin{lstlisting}
MAP
   NAME WCS_Server
   STATUS ON
   SIZE 400 300
   SYMBOLSET "../../etc/symbols.txt"
   EXTENT -2200000 -712631 3072800 3840000
   UNITS METERS
   SHAPEPATH "../data"
   IMAGECOLOR 255 255 255
   FONTSET "../../etc/fonts.txt"
END #Map file
\end{lstlisting}
\item Setelah itu kita masukan script untuk konfigurasi web interfacenya sebelum syntax END, didalam script ini berisi konfigurasi untuk web interface berupa tempat gambar serta metadata yang mendefinisakan informasi dalam tampilan antarmuka web.
\begin{lstlisting}
WEB
   IMAGEPATH "/ms4w/tmp/ms_tmp/"
   IMAGEURL "/ms_tmp/"
   METADATA
     "wcs_label"           "GMap WCS Demo Server" ### required
     "wcs_description"     "Some text description of the service"
     "wcs_onlineresource"  "http://localhost:8080/cgi-bin/mapserv.exe?map=C:/ms4w/apps/indo/map/mywcs.map&" 
     "wcs_fees"            "none"
     "wcs_accessconstraints"    "none"
     "wcs_keywordlist"          "wcs,test"
     "wcs_metadatalink_type"    "TC211"
     "wcs_metadatalink_format"  "text/plain"
     "wcs_metadatalink_href"    "http://awangga.net"
     "wcs_address"              "124 Gilmour Street"
     "wcs_city"                 "Ottawa"
     "wcs_stateorprovince"      "ON"
     "wcs_postcode"             "90210"
     "wcs_country"              "Canada"
     "wcs_contactelectronicmailaddress" "blah@blah"
     "wcs_contactperson"            "me"
     "wcs_contactorganization"      "unemployed"
     "wcs_contactposition"          "manager"
     "wcs_contactvoicetelephone"    "613-555-1234"
     "wcs_contactfacimiletelephone" "613-555-1235"
     "wcs_service_onlineresource"   "http://localhost:8080/cgi-bin/mapserv.exe?map=C:/ms4w/apps/indo/map/mywcs.map&"
     "wcs_enable_request"           "*"
   END #Metadata
 END #Web
\end{lstlisting}

\item Lalu kita masukan script projection untuk menentukan sistem referensi geografis agar map dapar memberikan output sesuai dengan source mapnya.
\begin{lstlisting}
PROJECTION
     "init=epsg:4326"
END
\end{lstlisting}

\item Setelah itu masukan script layer yang sangat dibutuhkan agar map dapat ditampilkan, disini kita membuat meta data juga untuk memberikan informasi seperti label dan range serta range label. tidak lupa disini juga mendefinisikan file map yang akan di tampilkan pada layer ini serta tipe mapfile pada layer ini.
\begin{lstlisting}
LAYER
   NAME bathymetry
   METADATA
     "wcs_label"           "Elevation/Bathymetry"  ### required
     "wcs_rangeset_name"   "Range 1"  ### required to support DescribeCoverage request
     "wcs_rangeset_label"  "My Label" ### required to support DescribeCoverage request
   END
   TYPE RASTER ### required
   STATUS ON
   DATA Indonesia_DNI_poster-map_1200x800mm-300dpi_v20170512.tif
 END
\end{lstlisting}

\item Berikut script lengkap untuk file wcs.map ini
\begin{lstlisting}
MAP
 NAME WCS_server
 STATUS ON
 SIZE 400 300
 SYMBOLSET "../../etc/symbols.txt"
 EXTENT -2200000 -712631 3072800 3840000
 UNITS METERS
 SHAPEPATH "../data"
 IMAGECOLOR 255 255 255
 FONTSET "../../etc/fonts.txt"
 #
 # Start of web interface definition
 #
 WEB
   IMAGEPATH "/ms4w/tmp/ms_tmp/"
   IMAGEURL "/ms_tmp/"
   METADATA
     "wcs_label"           "GMap WCS Demo Server" ### required
     "wcs_description"     "Some text description of the service"
     "wcs_onlineresource"  "http://localhost:8080/cgi-bin/mapserv.exe?map=C:/ms4w/apps/indo/map/mywcs.map&" ### recommended
     "wcs_fees"            "none"
     "wcs_accessconstraints"    "none"
     "wcs_keywordlist"          "wcs,test"
     "wcs_metadatalink_type"    "TC211"
     "wcs_metadatalink_format"  "text/plain"
     "wcs_metadatalink_href"    "http://awangga.net"
     "wcs_address"              "124 Gilmour Street"
     "wcs_city"                 "Ottawa"
     "wcs_stateorprovince"      "ON"
     "wcs_postcode"             "90210"
     "wcs_country"              "Canada"
     "wcs_contactelectronicmailaddress" "blah@blah"
     "wcs_contactperson"            "me"
     "wcs_contactorganization"      "unemployed"
     "wcs_contactposition"          "manager"
     "wcs_contactvoicetelephone"    "613-555-1234"
     "wcs_contactfacimiletelephone" "613-555-1235"
     "wcs_service_onlineresource"   "http://localhost:8080/cgi-bin/mapserv.exe?map=C:/ms4w/apps/indo/map/mywcs.map&"
     "wcs_enable_request"           "*"
   END
 END

 PROJECTION
   "init=epsg:4326"
 END

 LAYER
   NAME bathymetry
   METADATA
     "wcs_label"           "Elevation/Bathymetry"  ### required
     "wcs_rangeset_name"   "Range 1"  ### required to support DescribeCoverage request
     "wcs_rangeset_label"  "My Label" ### required to support DescribeCoverage request
   END
   TYPE RASTER ### required
   STATUS ON
   DATA Indonesia_DNI_poster-map_1200x800mm-300dpi_v20170512.tif
 END
END # Map File
\end{lstlisting}

\item Untuk memeriksa hasilnya berikut linknya untuk di buka di web browser:
\begin{lstlisting}
http://localhost:8080/cgi-bin/mapserv.exe?map=C:/ms4w/apps/indo/map/mywcs.map&SERVICE=WCS&VERSION=1.0.0&REQUEST=GetCapabilities
\end{lstlisting}

Jangan lupa untuk merubah direktori map dan parameter service berisi WCS, hasilnya berupa script xml.
\end{enumerate}

\subsection{Membuat Mapfile untuk WFS}
\begin{enumerate}
\item Buat file baru dengan nama wfs.map untuk menyiman script konfigurasi wfs
\item Masukan script map objek di bawah ini ke file wfs.map, pada script ini kita menentukan lebar peta, warna gambar serta mendefinisikan sumber map, fontset dan symbolset yang akan digunakan. 

\begin{lstlisting}
MAP
 NAME "WFS_server"
 STATUS ON
 SIZE 400 300
 #SYMBOLSET "../../etc/symbols.txt"
 EXTENT -180 -90 180 90
 UNITS DD
 SHAPEPATH "../data"
 IMAGECOLOR 255 255 255
 #FONTSET "../../etc/fonts.txt"
END #Map file
\end{lstlisting}
\item Setelah itu kita masukan script untuk konfigurasi web interfacenya sebelum syntax END, didalam script ini berisi konfigurasi untuk web interface berupa tempat gambar serta metadata yang mendefinisakan informasi dalam tampilan antarmuka web.

\begin{lstlisting}
 WEB
   IMAGEPATH "ms4w/tmp/ms_tmp/"
   IMAGEURL "/ms_tmp/"
   METADATA
     "wfs_title"                 "WFS Demo Server for MapServer" ## REQUIRED
     "wfs_onlineresource" "http://localhost:8080/cgi-bin/mapserv.exe?map=C:/ms4w/apps/indo/map/mywfs.map" ## Recommended      
     "wfs_srs"                    "EPSG:4326 EPSG:4269 EPSG:3978 EPSG:3857" ## Recommended
     "wfs_abstract"            "This text describes my WFS service." ## Recom
     "wfs_enable_request" "*"  # necessary
   END
 END
\end{lstlisting}

\item Lalu kita masukan script projection  untuk menentukan sistem referensi geografis agar map dapar memberikan output sesuai dengan source mapnya.

\begin{lstlisting}
 PROJECTION
     "init=epsg:4326"
 END
\end{lstlisting}

\item Tambahkan dan aktifkan syntax debug untuk mempermudah debuging jika terdapat error ketika mapfile dibuka/dijalankan.
\begin{lstlisting}
DEBUG on
\end{lstlisting}

\item Tambahkan script output format untuk mengatur format gambar yang di keluarkan pada mapfile ini, ada 4 jenis output yaitu png, png8, png256, jpg.
\begin{lstlisting}
 OUTPUTFORMAT
   NAME "png"
   DRIVER "AGG/PNG"
   IMAGEMODE "rgba"
   EXTENSION "png"
   MIMETYPE "image/png"
   IMAGEMODE RGBA
 END

 OUTPUTFORMAT
   NAME "png8"
   DRIVER "AGG/PNG"
   IMAGEMODE "rgba"
   EXTENSION "png"
   MIMETYPE "image/png"
   IMAGEMODE RGBA
   TRANSPARENT ON
   FORMATOPTION "QUANTIZE_FORCE=ON"
   FORMATOPTION "QUANTIZE_DITHER=ON"
   FORMATOPTION "QUANTIZE_COLORS=250"
 END

  OUTPUTFORMAT
   NAME "png256"
   DRIVER "AGG/PNG"
   IMAGEMODE "pc256"
   EXTENSION "png"
 END

 OUTPUTFORMAT
   NAME "jpg"
   DRIVER "AGG/JPEG"
   EXTENSION "jpg"
   FORMATOPTION "QUALITY=85"
 END
\end{lstlisting}
\item Setelah itu masukan script layer untuk menampilkan data provinsi, disini kita membuat class untuk setiap provinsi. tidak lupa disini juga mendefinisikan file map yang akan ditampilkan pada layer ini serta tipe mapfile pada layer ini. 

\begin{lstlisting}
LAYER #provinsi layer
   NAME provinsi
    MINSCALE 500000
   GROUP roads
   TYPE POLYGON
   DUMP true
   #transparency alpha
   STATUS on
   DATA "indonesia/00"
   LABELITEM "PROVINSI"
   CLASSITEM "PROVINSI"
   CLASS
     EXPRESSION "ACEH"
     STYLE
       COLOR 153 255 153
     END
     LABEL
       COLOR  0 0 0
       OUTLINECOLOR 255 255 255
       FONT "FreeSans"
       TYPE truetype
       SIZE 6
       POSITION lc
       PARTIALS true
       MINDISTANCE 200
     END
   END
   CLASS
     EXPRESSION "SUMATERA UTARA"
     STYLE
       COLOR 204 255 204
     END
     LABEL
       COLOR  0 0 0
       OUTLINECOLOR 255 255 255
       FONT "FreeSans"
       TYPE truetype
       SIZE 6
       POSITION lc
       PARTIALS true
       MINDISTANCE 200
     END
   END
   CLASS
     EXPRESSION "SUMATERA BARAT"
     STYLE
       COLOR 229 255 204
     END
     LABEL
       COLOR  0 0 0
       OUTLINECOLOR 255 255 255
       FONT "FreeSans"
       TYPE truetype
       SIZE 6
       POSITION lc
       PARTIALS true
       MINDISTANCE 200
     END
   END
   CLASS
     EXPRESSION "RIAU"
     STYLE
       COLOR 204 255 229
     END
     LABEL
       COLOR  0 0 0
       OUTLINECOLOR 255 255 255
       FONT "FreeSans"
       TYPE truetype
       SIZE 6
       POSITION lc
       PARTIALS true
       MINDISTANCE 200
     END
   END
   CLASS
     EXPRESSION "JAMBI"
     STYLE
       COLOR 153 255 204
     END
     LABEL
       COLOR  0 0 0
       OUTLINECOLOR 255 255 255
       FONT "FreeSans"
       TYPE truetype
       SIZE 6
       POSITION lc
       PARTIALS true
       MINDISTANCE 200
     END
   END
   CLASS
     EXPRESSION "SUMATERA SELATAN"
     STYLE
       COLOR 153 255 153
     END
     LABEL
       COLOR  0 0 0
       OUTLINECOLOR 255 255 255
       FONT "FreeSans"
       TYPE truetype
       SIZE 6
       POSITION lc
       PARTIALS true
       MINDISTANCE 200
     END
   END
   CLASS
     EXPRESSION "BENGKULU"
     STYLE
       COLOR 204 255 204
     END
     LABEL
       COLOR  0 0 0
       OUTLINECOLOR 255 255 255
       FONT "FreeSans"
       TYPE truetype
       SIZE 6
       POSITION lc
       PARTIALS true
       MINDISTANCE 200
     END
   END
   CLASS
     EXPRESSION "LAMPUNG"
     STYLE
       COLOR 229 255 204
     END
     LABEL
       COLOR  0 0 0
       OUTLINECOLOR 255 255 255
       FONT "FreeSans"
       TYPE truetype
       SIZE 6
       POSITION lc
       PARTIALS true
       MINDISTANCE 200
     END
   END
   CLASS
     EXPRESSION "KEPULAUAN BANGKA BELITUNG"
     STYLE
       COLOR 204 255 229
     END
     LABEL
       COLOR  0 0 0
       OUTLINECOLOR 255 255 255
       FONT "FreeSans"
       TYPE truetype
       SIZE 6
       POSITION lc
       PARTIALS true
       MINDISTANCE 200
     END
   END
   CLASS
     EXPRESSION "KEPULAUAN RIAU"
     STYLE
       COLOR 153 255 204
     END
     LABEL
       COLOR  0 0 0
       OUTLINECOLOR 255 255 255
       FONT "FreeSans"
       TYPE truetype
       SIZE 6
       POSITION lc
       PARTIALS true
       MINDISTANCE 200
     END
   END
   CLASS
     EXPRESSION "DKI JAKARTA"
     STYLE
       COLOR 0 255 0
     END
     LABEL
       COLOR  0 0 0
       OUTLINECOLOR 255 255 255
       FONT "FreeSans"
       TYPE truetype
       SIZE 6
       POSITION lc
       PARTIALS true
       MINDISTANCE 200
     END
   END
   CLASS
     EXPRESSION "JAWA BARAT"
     STYLE
       COLOR 204 255 204
     END
     LABEL
       COLOR  0 0 0
       OUTLINECOLOR 255 255 255
       FONT "FreeSans"
       TYPE truetype
       SIZE 6
       POSITION lc
       PARTIALS true
       MINDISTANCE 200
     END
   END
   CLASS
     EXPRESSION "JAWA TENGAH"
     STYLE
       COLOR 229 255 204
     END
     LABEL
       COLOR  0 0 0
       OUTLINECOLOR 255 255 255
       FONT "FreeSans"
       TYPE truetype
       SIZE 6
       POSITION lc
       PARTIALS true
       MINDISTANCE 200
     END
   END
   CLASS
     EXPRESSION "DAERAH ISTIMEWA YOGYAKARTA"
     STYLE
       COLOR 204 255 229
     END
     LABEL
       COLOR  0 0 0
       OUTLINECOLOR 255 255 255
       FONT "FreeSans"
       TYPE truetype
       SIZE 6
       POSITION lc
       PARTIALS true
       MINDISTANCE 200
     END
   END
   CLASS
     EXPRESSION "JAWA TIMUR"
     STYLE
       COLOR 153 255 204
     END
     LABEL
       COLOR  0 0 0
       OUTLINECOLOR 255 255 255
       FONT "FreeSans"
       TYPE truetype
       SIZE 6
       POSITION lc
       PARTIALS true
       MINDISTANCE 200
     END
   END
   CLASS
     EXPRESSION "BANTEN"
     STYLE
       COLOR 153 255 153
     END
     LABEL
       COLOR  0 0 0
       OUTLINECOLOR 255 255 255
       FONT "FreeSans"
       TYPE truetype
       SIZE 6
       POSITION lc
       PARTIALS true
       MINDISTANCE 200
     END
   END
   CLASS
     EXPRESSION "BALI"
     STYLE
       COLOR 204 255 204
     END
     LABEL
       COLOR  0 0 0
       OUTLINECOLOR 255 255 255
       FONT "FreeSans"
       TYPE truetype
       SIZE 6
       POSITION lc
       PARTIALS true
       MINDISTANCE 200
     END
   END
   CLASS
     EXPRESSION "NUSA TENGGARA BARAT"
     STYLE
       COLOR 229 255 204
     END
     LABEL
       COLOR  0 0 0
       OUTLINECOLOR 255 255 255
       FONT "FreeSans"
       TYPE truetype
       SIZE 6
       POSITION lc
       PARTIALS true
       MINDISTANCE 200
     END
   END
   CLASS
     EXPRESSION "NUSA TENGGARA TIMUR"
     STYLE
       COLOR 204 255 229
     END
     LABEL
       COLOR  0 0 0
       OUTLINECOLOR 255 255 255
       FONT "FreeSans"
       TYPE truetype
       SIZE 6
       POSITION lc
       PARTIALS true
       MINDISTANCE 200
     END
   END
   CLASS
     EXPRESSION "KALIMANTAN BARAT"
     STYLE
       COLOR 153 255 204
     END
     LABEL
       COLOR  0 0 0
       OUTLINECOLOR 255 255 255
       FONT "FreeSans"
       TYPE truetype
       SIZE 6
       POSITION lc
       PARTIALS true
       MINDISTANCE 200
     END
   END
   CLASS
     EXPRESSION "KALIMANTAN TENGAH"
     STYLE
       COLOR 153 255 153
     END
     LABEL
       COLOR  0 0 0
       OUTLINECOLOR 255 255 255
       FONT "FreeSans"
       TYPE truetype
       SIZE 6
       POSITION lc
       PARTIALS true
       MINDISTANCE 200
     END
   END
   CLASS
     EXPRESSION "KALIMANTAN SELATAN"
     STYLE
       COLOR 204 255 204
     END
     LABEL
       COLOR  0 0 0
       OUTLINECOLOR 255 255 255
       FONT "FreeSans"
       TYPE truetype
       SIZE 6
       POSITION lc
       PARTIALS true
       MINDISTANCE 200
     END
   END
   CLASS
     EXPRESSION "KALIMANTAN TIMUR"
     STYLE
       COLOR 229 255 204
     END
     LABEL
       COLOR  0 0 0
       OUTLINECOLOR 255 255 255
       FONT "FreeSans"
       TYPE truetype
       SIZE 6
       POSITION lc
       PARTIALS true
       MINDISTANCE 200
     END
   END
   CLASS
     EXPRESSION "KALIMANTAN UTARA"
     STYLE
       COLOR 204 255 229
     END
     LABEL
       COLOR  0 0 0
       OUTLINECOLOR 255 255 255
       FONT "FreeSans"
       TYPE truetype
       SIZE 6
       POSITION lc
       PARTIALS true
       MINDISTANCE 200
     END
   END
   CLASS
     EXPRESSION "SULAWESI UTARA"
     STYLE
       COLOR 153 255 204
     END
     LABEL
       COLOR  0 0 0
       OUTLINECOLOR 255 255 255
       FONT "FreeSans"
       TYPE truetype
       SIZE 6
       POSITION lc
       PARTIALS true
       MINDISTANCE 200
     END
   END
   CLASS
     EXPRESSION "SULAWESI TENGAH"
     STYLE
       COLOR 153 255 153
     END
     LABEL
       COLOR  0 0 0
       OUTLINECOLOR 255 255 255
       FONT "FreeSans"
       TYPE truetype
       SIZE 6
       POSITION lc
       PARTIALS true
       MINDISTANCE 200
     END
   END
   CLASS
     EXPRESSION "SULAWESI SELATAN"
     STYLE
       COLOR 204 255 204
     END
     LABEL
       COLOR  0 0 0
       OUTLINECOLOR 255 255 255
       FONT "FreeSans"
       TYPE truetype
       SIZE 6
       POSITION lc
       PARTIALS true
       MINDISTANCE 200
     END
   END
   CLASS
     EXPRESSION "SULAWESI TENGGARA"
     STYLE
       COLOR 229 255 204
     END
     LABEL
       COLOR  0 0 0
       OUTLINECOLOR 255 255 255
       FONT "FreeSans"
       TYPE truetype
       SIZE 6
       POSITION lc
       PARTIALS true
       MINDISTANCE 200
     END
   END
   CLASS
     EXPRESSION "GORONTALO"
     STYLE
       COLOR 204 255 229
     END
     LABEL
       COLOR  0 0 0
       OUTLINECOLOR 255 255 255
       FONT "FreeSans"
       TYPE truetype
       SIZE 6
       POSITION lc
       PARTIALS true
       MINDISTANCE 200
     END
   END
   CLASS
     EXPRESSION "SULAWESI BARAT"
     STYLE
       COLOR 153 255 204
     END
     LABEL
       COLOR  0 0 0
       OUTLINECOLOR 255 255 255
       FONT "FreeSans"
       TYPE truetype
       SIZE 6
       POSITION lc
       PARTIALS true
       MINDISTANCE 200
     END
   END
   CLASS
     EXPRESSION "MALUKU"
     STYLE
       COLOR 153 255 153
     END
     LABEL
       COLOR  0 0 0
       OUTLINECOLOR 255 255 255
       FONT "FreeSans"
       TYPE truetype
       SIZE 6
       POSITION lc
       PARTIALS true
       MINDISTANCE 200
     END
   END
   CLASS
     EXPRESSION "MALUKU UTARA"
     STYLE
       COLOR 204 255 204
     END
     LABEL
       COLOR  0 0 0
       OUTLINECOLOR 255 255 255
       FONT "FreeSans"
       TYPE truetype
       SIZE 6
       POSITION lc
       PARTIALS true
       MINDISTANCE 200
     END
   END
   CLASS
     EXPRESSION "PAPUA BARAT"
     STYLE
       COLOR 229 255 204
     END
     LABEL
       COLOR  0 0 0
       OUTLINECOLOR 255 255 255
       FONT "FreeSans"
       TYPE truetype
       SIZE 6
       POSITION lc
       PARTIALS true
       MINDISTANCE 200
     END
   END
   CLASS
     EXPRESSION "PAPUA"
     STYLE
       COLOR 204 255 229
     END
     LABEL
       COLOR  0 0 0
       OUTLINECOLOR 255 255 255
       FONT "FreeSans"
       TYPE truetype
       SIZE 6
       POSITION lc
       PARTIALS true
       MINDISTANCE 200
     END
   END
 END #layer provinsi
\end{lstlisting}

\item Selain layer untuk provinsi kita tambahkan juga layer untuk kabupaten kota berikut ini
\begin{lstlisting}
 LAYER
   NAME base_map
    GROUP roads
   TYPE POLYGON
   STATUS ON
   DATA "indonesia/00"
   POSTLABELCACHE FALSE
   PROCESSING "LABEL_NO_CLIP=ON"
   LABELCACHE ON
   LABELITEM "KABKOT"
   CLASS
     #minscale 10000
     maxscale 1500000
     NAME "indonesia_kab"
     STYLE
       COLOR 102 255 102
       OUTLINECOLOR 200 200 200
       SYMBOL 0
     END
     LABEL
       COLOR  0 0 0
       OUTLINECOLOR 255 255 255
       FONT "FreeSans"
       TYPE truetype
       SIZE 8
       POSITION CC
       PARTIALS TRUE
       MINDISTANCE 50
       REPEATDISTANCE 9999
     END
   END
   METADATA
     "DESCRIPTION"   "Peta Indonesia"
   END
 END #layer kabkot indonesia
\end{lstlisting}
\end{enumerate}

\subitem Untuk memeriksa hasilnya berikut linknya untuk di buka di web browser:
\begin{lstlisting}
http://localhost:8080/cgi-bin/mapserv.exe?map=map/kampus.map&VERSION=1.1.1&REQUEST=GetMap&LAYERS=roads&STYLES=&SRS=EPSG:4326&BBOX=94.5011475,-11.007385,141.01947,6.076721&WIDTH=1024&HEIGHT=768&FORMAT=image/png
\end{lstlisting}
Jangan lupa untuk merubah direktori map dan parameter service berisi WMS, hasilnya berupa gambar peta.