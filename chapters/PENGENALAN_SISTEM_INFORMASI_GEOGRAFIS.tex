\section{Definisi GIS} 
Sistem Informasi Geografis atau disingkat SIG dalam bahasa Inggris Geographic Information System (disingkat GIS) merupakan sistem informasi khusus yang mengelola data yang memiliki informasi spasial (bereferensi keruangan). Atau dalam arti yang lebih sempit adalah sistem komputer yang memiliki kemampuan untuk membangun, menyimpan, mengelola dan menampilkan informasi bereferensi geografis atau data geospasial untuk mendukung pengambilan keputusan dalam perencanaan dan pengelolaan suatu wilayah, misalnya data yang diidentifikasi menurut lokasinya, dalam sebuah database. Para praktisi juga memasukkan orang yang membangun dan mengoperasikannya dan data sebagai bagian dari sistem ini. 
Teknologi Sistem Informasi Geografis dapat digunakan untuk investigasi ilmiah, pengelolaan sumber daya, perencanaan pembangunan, kartografi dan perencanaan rute. Misalnya, SIG bisa membantu perencana untuk secara cepat menghitung waktu tanggap darurat saat terjadi bencana alam, atau SIG dapat digunaan untuk mencari lahan basah (wetlands) yang membutuhkan perlindungan dari polusi atau dapat digunakan mencari informasi sebuah tempat khusus dan banyak manfaat lain yang dapat ikembangkan dalam sistem informasi geografis ini. 
