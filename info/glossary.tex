\term{Wetlands}Adalah suatu wilayah daratan yang digenangi air atau memiliki kandungan air yang cukup tinggi.

\term{Geografis}Adalah letak suatu wilayah yang dapat dilihat dari kenyataan di permukaan bumi atau posisi wilayah tersebut dapat di lihat pada bola bumi.

\term{Topografi}Merupakan peta yang menggambarkan bentuk relief yakni tinggi rendahnya permukaan Bumi.
 
\term{Tematik}Merupakan peta yang hanya menampilkan sebagian permukaan bumi, dan peta tematik dapat menyajikan tema tertentu dan kepentingan tertentu seperti (status,penduduk,tansportasi). 

\term{Storage}Adalah sebuah perangkat digital yang berfungsi untuk menyimpan berbagai macam data digital yang dapat di simpan dalam kurun waktu yang tidak menentu tergantung usia dan perawatan dari perangkat Storage itu sendiri.

\term{Overlay}Penggabungan dua data atau lebih secara tumpang susun atau secara tepat untuk memperoleh data grafis baru yang memiliki satuan pemetaan.

\term{Buffer Zone}Lahan yang tidak dibangun dan dibiarkan sebagaimana aslinya seperti (Hutan, danau, sungai).

\term{Device}Adalah perangkat komputer yang berfungsi untuk memasukkan data atau perintah ke dalam komputer berupa teks, grafik, gambar, suara, dll.

\term{Vector}Merupakan sebuah gambar yang terbentuk dari sejumlah garis/kurva.

\term{Raster}Merupakan gambar yang terbentuk dari titik - titik atau piksel

\term{Pixel}Adalah kumpulan titik titik yang berwarna yang berdekatan sehingga terlihat membentuk sebuah gambar.

\term{Polygon}Merupakan serangkaian titik-titik yang dihubungkan dengan garis lurus sehingga membentuk sebuah rangkaian gambar.
