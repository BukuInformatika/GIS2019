\section{Definisi GIS} 
Sistem Informasi Geografis atau disingkat SIG dalam bahasa Inggris Geographic Information System (disingkat GIS) merupakan sistem informasi khusus yang mengelola data yang memiliki informasi spasial (bereferensi keruangan). Atau dalam arti yang lebih sempit adalah sistem komputer yang memiliki kemampuan untuk membangun, menyimpan, mengelola dan menampilkan informasi bereferensi geografis atau data geospasial untuk mendukung pengambilan keputusan dalam perencanaan dan pengelolaan suatu wilayah, misalnya data yang diidentifikasi menurut lokasinya, dalam sebuah database. Para praktisi juga memasukkan orang yang membangun dan mengoperasikannya dan data sebagai bagian dari sistem ini. 
Teknologi Sistem Informasi Geografis dapat digunakan untuk investigasi ilmiah, pengelolaan sumber daya, perencanaan pembangunan, kartografi dan perencanaan rute. Misalnya, SIG bisa membantu perencana untuk secara cepat menghitung waktu tanggap darurat saat terjadi bencana alam, atau SIG dapat digunaan untuk mencari lahan basah (wetlands) yang membutuhkan perlindungan dari polusi atau dapat digunakan mencari informasi sebuah tempat khusus dan banyak manfaat lain yang dapat ikembangkan dalam sistem informasi geografis ini. 

\subsection{Pengertian GIS Menurut Para Ahli}
\begin{enumerate}

\item \textbf{Aronaff (1989)}
SIG adalah sistem informasi yang didasarkan pada kerja komputer yang memasukkan, mengelola, memanipulasi dan menganalisis data serta memberi uraian.

\item \textbf{Burrough (1986)}
SIG merupakan alat yang bermanfaat untuk pengumpulan, penimbunan, pengambilan kembali data yang diinginkan dan penayangan data keruangan yang berasal dari kenyataan dunia.

\item \textbf{Murai (1999)}
SIG sebagai sistem informasi yang digunakan untuk memasukkan, menyimpan, memanggil kembali, mengolah, menganalisis dan menghasilkan data bereferensi geografis atau data geospasial untuk mendukung pengambilan keputusan dalam perencanaan dan pengelolaan penggunaan lahan, sumber daya alam, lingkungan, transportasi, fasilitas kota, dan pelayanan umum lainnya.

\item \textbf{Marble et al (1983)}
SIG merupakan sistem penanganan data keruangan.  Bernhardsen (2002) SIG sebagai sistem komputer ang digunakan untuk memanipulasi data geografi. Sistem ini diimplementasikan dengan perangkat keras dan perangkat lunak komputer yang berfungsi untuk akusisi dan verifikasi data, kompilasi data, penyimpanan data, perubahan dan pembaharuan data, manajemen dan pertukaran data, manipulasi data, pemanggilan dan presentasi data serta analisa data.

\item \textbf{Gistut (1994)}
SIG adalah sistem yang dapat mendukung pengambilan keputusan spasial dan mampu mengintegrasikan deskripsi-deskripsi lokasi dengan karakteristik-karakteristik fenomena yang ditemukan di lokasi tersebut. SIG yang lengkap mencakup metodologi dan teknologi yang diperlukan, yaitu data spasial perangkat keras, perangkat lunak dan struktur organisasi  Berry (1988) SIG merupakan sistem informasi, referensi internal, serta otomatisasi data keruangan.

\item \textbf{Calkin dan Tomlison (1984)}
SIG merupakan sistem komputerisasi data yang penting.  Linden, (1987) SIG adalah sistem untuk pengelolaan, penyimpanan, pemrosesan (manipulasi), analisis dan penayangan data secara spasial terkait dengan muka bumi.
\end{enumerate}